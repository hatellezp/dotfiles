

% This is a simple sample document.  For more complicated documents take a look in the exercise tab. Note that everything that comes after a % symbol is treated as comment and ignored when the code is compiled.

\documentclass{article} % \documentclass{} is the first command in any LaTeX code.  It is used to define what kind of document you are creating such as an article or a book, and begins the document preamble

\usepackage{amsmath} % \usepackage is a command that allows you to add functionality to your LaTeX code

\title{Latent Space Translation} % Sets article title
\author{Horacio Tellez} % Sets authors name
\date{\today} % Sets date for date compiled

% The preamble ends with the command \begin{document}
\begin{document} % All begin commands must be paired with an end command somewhere
    \maketitle % creates title using information in preamble (title, author, date)


This is about closing the gap between real and simulated data using a latent 
space as means of the translation. This should be seen as a three
step:
\begin{enumerate}
    \item Go into the latent space with a \emph{simulated signal}, this
        defines a \emph{latent simulated signal}; 
    \item Go from the \emph{latent simulated signal} to a \emph{real simulated signal};
    \item Go from the \emph{real simulated signal} to a \emph{real signal}.
\end{enumerate}

We have implicitly assumed defined (at least three objects):
\begin{itemize}
    \item A map from the signal space to the latent space (call it the encoder $E$);
    \item A map from the latent space into itself (the actual translation between simulated and real data,
        let's call it the translator $M$);
    \item A map from the latent space to the signal space (call it the decoder $D$).
\end{itemize}

\paragraph{Encoder-Decoder}
A good way to solve the auto-encoder problem in this case is a \emph{variational auto-encoder} (VAE). This will
create a encoder-decoder pair that is aware of the importance of features and where the latent space shares some
form of continuity with the signal space.

\textbf{How to train a VAE}
This is only dependent of the data, no need to be aware of type of data, simulated or real. Even if conditioning 
is possible, better to test first without.

For the actual method of training, the literature for VAE is very vast.

\paragraph{Translator}

The translator is the complex part of this setting.

Let $X = (f_1, \dots, f_n, d)$ be an real signal, where $f_1, \dots, f_n$ are known an informative features as
\begin{itemize}
    \item type of object: cylinder, sphere, mine, shoe, rock, fish, \ldots
    \item physical characteristics (if it applies): radius, length, height, width, solid or not \ldots
    \item spatial situation: seabed, in the soil, in free space, \ldots,
    \item \ldots 
\end{itemize}
and $d$ is the actual signal.
What is mapped to the latent space is only the signal: $d_l = E(d)$. Now suppose that for such a signal $X$ we
can simulate a signal $Y$ with not necessarily the same features, but the same signal: $Y = (f'_1, \dots, f'_n, d)$.

\end{document} % This is the end of the document
